%%% PREAMBLE - Do not touch %%%%%%%%%%%%%%%%%%%%%%%%%%%%%%%%%%%%%%%%%%%%%%%%%%%%%%
\documentclass[10pt,twocolumn,letterpaper]{article}
\usepackage[ansinew]{inputenc}
\usepackage[portuges,brazil,english]{babel}
\usepackage{model}
\usepackage{times}
\usepackage{epsfig}
\usepackage{graphicx}
\usepackage{amsmath}
\usepackage{amssymb}
\usepackage{color}
\usepackage[pagebackref=true,breaklinks=true,letterpaper=true,colorlinks,bookmarks=false]{hyperref}
\input{pics/abaco}

\cvprfinalcopy % *** Uncomment this line for the final submission
\def\httilde{\mbox{\tt\raisebox{-.5ex}{\symbol{126}}}}
\ifcvprfinal\pagestyle{empty}\fi

\newcommand{\TODO}[1]{TODO: #1}
\newcommand{\CITEONE}[2]{\mbox{#1 \cite{#2}}}
\newcommand{\CITETWO}[3]{\mbox{#1 and #2 \cite{#3}}}
\newcommand{\CITEN}[2]{\mbox{#1 et al. \cite{#2}}}

%%% Paper beginning %%%%%%%%%%%%%%%%%%%%%%%%%%%%%%%%%%%%%%%%%%%%%%%%%%%%%%%%%%%%%%
\begin{document}

%%% Title and authors %%%%%%%%%%%%%%%%%%%%%%%%%%%%%%%%%%%%%%%%%%%%%%%%%%%%%%%%%%%%
\title{Here goes the paper title}
\author{Fulano da Silva\thanks{Is with the Institute of Computing, University of Campinas (Unicamp). \textbf{Contact}: \tt\small{fulano@ic.unicamp.br}}\\
Fulano da Silva\thanks{Is with the Institute of Computing, University of Campinas (Unicamp). \textbf{Contact}: \tt\small{fulano@ic.unicamp.br}}\\
Anderson Rocha\thanks{Is with the Institute of Computing, University of Campinas (Unicamp). \textbf{Contact}: \tt\small{anderson.rocha@ic.unicamp.br}}
}

%%% Abstract %%%%%%%%%%%%%%%%%%%%%%%%%%%%%%%%%%%%%%%%%%%%%%%%%%%%%%%%%%%%%%%%%%%%%
\maketitle
\begin{abstract}
Here goes the abstract. 
\end{abstract}

%%% Introduction %%%%%%%%%%%%%%%%%%%%%%%%%%%%%%%%%%%%%%%%%%%%%%%%%%%%%%%%%%%%%%%%%
\section{Introduction}
Here goes the introduction and motivation of the work.

Some directions for the paper:

\begin{itemize}
	\item Diagrams and figures are encouraged for making the paper richer
	\item The sections proposed here are not hard-constrained. It means, you 
	can propose other sections as well as change the existing ones. 
\end{itemize}

%%% Add section %%%%%%%%%%%%%%%%%%%%%%%%%%%%%%%%%%%%%%%%%%%%%%%%%%%%%%%%%%%%%%%%%%
\section{State-of-the-Art}
Here goes the state-of-the-art research (talk about prior work for solving the same problem). 

%%% Add section %%%%%%%%%%%%%%%%%%%%%%%%%%%%%%%%%%%%%%%%%%%%%%%%%%%%%%%%%%%%%%%%%%
\section{Proposed Solution}
Talk about the proposed solution for the selected problem. 

%%% Add section %%%%%%%%%%%%%%%%%%%%%%%%%%%%%%%%%%%%%%%%%%%%%%%%%%%%%%%%%%%%%%%%%%
\section{Experiments and Discussion}
Talk about the experiments carried out and the obtained results. 

Examples of citations~\cite{Ni_2008, Ni_2009}. For direct citations use something like: 

\CITEONE{Silva}{Silva_2010} for papers with one author.
\CITETWO{Silva}{Souza}{Silva_2010b} for papers with two authors.
\CITEN{Silva}{Silva_2010c} for papers with three or more authors.

Example of a figure of one column. 
\begin{figure}
\begin{center}
	\includegraphics[width=0.99\columnwidth]{pics/example-figure}
	\caption{A figure example spanning one column only.\label{fig:label}}   
\end{center} 
\end{figure}   

Example of a figure spanning two columns. 
\begin{figure*}
\begin{center}
	\includegraphics[width=0.99\textwidth]{pics/example-figure-spanned}
	\caption{A figure example spanning two columns.\label{fig:label2}}   
\end{center} 
\end{figure*}

Example of a table spanning only one column: 

\begin{table}
\begin{center}
\begin{tabular}{l*{6}{c}r}
Team              & P & W & D & L & F  & A & Pts \\
\hline
Manchester United & 6 & 4 & 0 & 2 & 10 & 5 & 12  \\
Celtic            & 6 & 3 & 0 & 3 &  8 & 9 &  9  \\
Benfica           & 6 & 2 & 1 & 3 &  7 & 8 &  7  \\
FC Copenhagen     & 6 & 2 & 1 & 2 &  5 & 8 &  7  \\
\end{tabular}
\end{center}
\end{table}

Example of a table spanning two columns: 

\begin{table*}
\begin{center}
    \begin{tabular}{ | l | l | l | p{8cm} |}
    \hline
    Day & Min Temp & Max Temp & Summary \\ \hline
    Monday & 11C & 22C & A clear day with lots of sunshine.  
    However, the strong breeze will bring down the temperatures. \\ \hline
    Tuesday & 9C & 19C & Cloudy with rain, across many northern regions. Clear spells
    across most of Scotland and Northern Ireland,
    but rain reaching the far northwest. \\ \hline
    Wednesday & 10C & 21C & Rain will still linger for the morning.
    Conditions will improve by early afternoon and continue
    throughout the evening. \\
    \hline
    \end{tabular}
\end{center}    
\end{table*}

%%% Add section %%%%%%%%%%%%%%%%%%%%%%%%%%%%%%%%%%%%%%%%%%%%%%%%%%%%%%%%%%%%%%%%%%
\section{Conclusions and Future Work}
Present the main conclusions of the work as well as some future directions for other people interested in continuing this work. 

%%% References %%%%%%%%%%%%%%%%%%%%%%%%%%%%%%%%%%%%%%%%%%%%%%%%%%%%%%%%%%%%%%%%%%%
{\small
\bibliographystyle{unsrt}
\bibliography{referencias-exemplo}
}

\end{document}